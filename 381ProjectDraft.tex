\documentclass[twoside,twocolumn]{article}
    \usepackage[a4paper, left=2cm, right=2cm]{geometry} % A4 paper size and thin margins
    \usepackage[sc]{mathpazo} % Use the Palatino font
    \usepackage[T1]{fontenc} % Use 8-bit encoding that has 256 glyphs
    \usepackage{microtype} % Slightly tweak font spacing for aesthetics
    \usepackage[english]{babel} % Language hyphenation and typographical rules
    \usepackage{booktabs} % Horizontal rules in tables
    \usepackage{enumitem} % Customized lists
    \usepackage[table,xcdraw]{xcolor}
    \usepackage[utf8]{inputenc} % Required for inputting international characters
    \usepackage{parskip}
    \usepackage{graphicx}
    \usepackage{hyperref}
    \usepackage{pdfpages}
    \usepackage{amsmath}
    \usepackage{esvect}
    \usepackage{listings}
    \usepackage{spverbatim}
    \usepackage[title]{appendix}
    \hypersetup{
        colorlinks=true,
        linkcolor=blue,
        filecolor=magenta,      
        urlcolor=cyan,
    }
    \urlstyle{same}
    \setlength{\parindent}{18pt}
    \setlist[itemize]{noitemsep} % Make itemize lists more compact
    \makeatletter
    \newcommand*{\rom}[1]{\expandafter\@slowromancap\romannumeral #1@}
    \g@addto@macro{\UrlBreaks}{\UrlOrds}
    \makeatother

    \title{\LARGE \bf
    Optimization of Assignments for Teaching Assistants at UW
    }
    
    \author{ \parbox{3 in}{\centering Chongyi Xu, Weifan Jiang \\
             University of Washington\\
             MATH 381 Project Draft\\
             {\tt\small chongyix@uw.edu}}
    }

    \begin{document}
    \maketitle

    %----------------------------------------------------------------------------------------
    %	ARTICLE CONTENTS
    %----------------------------------------------------------------------------------------
    \begin{abstract}

    For this project, we are focusing on develop a method to assign teaching-assistant candidates
    to different courses. Assignning candidates to their preferred teaching positions has bothered
    course coordinators at UW for a long time. Our purpose is to recommend our method of teaching-assistant
    assignments to solve this problem. We used simulated data to compare different methods that 
    we would use in the project.
        
    \end{abstract}

    \linespread{1.05} % Line spacing - Palatino needs more space between lines
    %------------------------------------------------
    \section{Introduction}
    The assignment of different teaching positions is a complicated work. The word "teaching position" 
    includes teaching assistants, graders and instructors here at UW. Each type of position has its 
    unique qualifications and requirments. Some positions require teaching, while other do not. 
    Most students are deemed to be certified to teach. Those whose first language is not English must 
    pass the SPEAK test to be certified. The position qualification does not only appear in different roles 
    but also in different courses. For instance, the instructor positions would mostly be restricted to 
    graduate students. Meanwhile, the teaching assistant positions could open to both undergraduate students and 
    graduate students.

    
    Furthermore, the UW introduction courses, such as CSE 142(Computer Programming I), are always popular 
    that there are over 700 students registered for around 50 sections each quarter. On the other hand, there are 
    also tiny-sized courses that designed for under 10 students. Therefore, the assignments of teaching positions 
    must satisfy the requirements for every single course each quarter such that everyone who registered for the course
    could have equal opportunity and fairly distributed teaching resources.   

    Additionally, during the process of assignments, the course coordinators, who are in charge of assigning student candidates
    to appropriate roles, must consider about the preference list for seperated categories submitted by the candidates. Taking
    the example of UW CSE TA application form, as the candidates apply for CSE TA positions, they needs to choose their preferences 
    ("prefer not" or"neutral" or "prefer") for 12 distinct categories
    \begin{itemize}
        \item AI and Robotics			
        \item Architecture			
        \item Computational Biology			
        \item Databases, Information Retrieval			
        \item Graphics, Vision, Animation			
        \item Hardware			
        \item Human-Computer Interaction			
        \item Introductory (CSE100,14x, 190)			
        \item Languages, Compilers, Software Engineering			
        \item Systems, Networks
        \item Theory			
        \item Uncategorize
    \end{itemize}
    And following the information provided by CSE department, they initialize the preference value for each course that candidates prefer, 
    are neutral to, or prefer not to TA to 0.8, 0.5, and 0.2, respectively. This helps "push" candidates assignment towards courses 
    in areas they prefer and away from courses in areas they do not prefer. Without choosing preference directly, candidates could establish
    their course preferences as well. If candidates choose to make up their own list, they would be asked to fill in a numerical number 
    between 0.0 and 1.0 that represents their preference to teach each courses that they are certified to TA. Besides preferences from 
    candidates side, instructors preferences should also be considered. Instructors would be asked to fill in a form of preferred students.
    %------------------------------------------------
    Our motivation of the project is from the interview with undergraduate TAs and graduate TAs about their teaching experience in early quarters.
    They complained that many times, even though they self-report their preferences, the situation that they got assigned to a course which 
    indicated as "less-preferred" is still common. Therefore, we would like to develop such method which put student candidates' preferences and
    professors' preferences at maximal consideration, while not producting unpractical assignments.

    %------------------------------------------------
    \section{Theoretical Background}
    \subsection{Stable Match Algorithm}
    In the field of computer science and mathematics, stable match problem or stable marraige problem states that that given N men and N women, 
    where each person has ranked all members of the opposite sex in order of preference, marry the men and women together such that there are no 
    2 people of opposite sex who would both rather have each other than their current partners. If there are no such people, all the marriages are “stable”.
    In 1962, D. Gale and L. S. Shapley, Brown University and the RAND Corporation, proved that, for any equal number of men and women, it is guaranteed
    that there is a stable matching. In their paper "College Admission and the Stability of Marriage", they defined the stability as following, an assignments
    of applications to colleges will be called unstable if there are two applicants $\alpha$ and $\beta$ who are assigned to colleges A and B, respectively,
    although $\beta$ prefers A to B and A prefers $\beta$ to $\alpha$. And they considered a stable assignment to be optimal if every applicant is at least 
    as well off under it as under any other stable assignments.
    
    Gale-Shapley Algorithm
    \begin{spverbatim}
INPUT: preference list for men and women
INITIALIZE matching set S to an empty set
WHILE (some woman w in W is still unmatched and hasn't proposed to every man in M)
    m <- first man on w's preference list to whom w has not yet proposed
    IF (m is unmatched)
        ADD pair (m, w) to S
    ELSE IF (m prefers w to existing pair w')
        REPLACE (m, w') with (m, w) in S
        FREE w'
    ELSE 
        w REJECT m

RETURN matching S
    \end{spverbatim}

    \section{Algorithm Implementation and Development}
    \subsection{Genreal Procedure}
    
    %------------------------------------------------

    \section{Computational Results}
    \subsection{Test 1}
        

    \section{Summary and Conclusions}
    
    %----------------------------------------------------------------------------------------
    %	APPENDIX
    %----------------------------------------------------------------------------------------

    \mbox{~}
    \clearpage
    \begin{appendices}
    \setboolean{@twoside}{false}
    \setboolean{@twocolumn}{false}
    \end{appendices}

\end{document}