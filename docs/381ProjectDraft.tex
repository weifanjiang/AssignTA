\documentclass[twoside,twocolumn]{article}
    \usepackage[a4paper, left=2cm, right=2cm]{geometry} % A4 paper size and thin margins
    \usepackage[sc]{mathpazo} % Use the Palatino font
    \usepackage[T1]{fontenc} % Use 8-bit encoding that has 256 glyphs
    \usepackage{microtype} % Slightly tweak font spacing for aesthetics
    \usepackage[english]{babel} % Language hyphenation and typographical rules
    \usepackage{booktabs} % Horizontal rules in tables
    \usepackage{enumitem} % Customized lists
    \usepackage[table,xcdraw]{xcolor}
    \usepackage[utf8]{inputenc} % Required for inputting international characters
    \usepackage{parskip}
    \usepackage{graphicx}
    \usepackage{hyperref}
    \usepackage{pdfpages}
    \usepackage{amsmath}
    \usepackage{esvect}
    \usepackage{listings}
    \usepackage{spverbatim}
    \usepackage[title]{appendix}
    \hypersetup{
        colorlinks=true,
        linkcolor=blue,
        filecolor=magenta,      
        urlcolor=cyan,
    }
    \lstset{
        basicstyle=\ttfamily,
        frame=single
    }
    \urlstyle{same}
    \setlength{\parindent}{18pt}
    \setlist[itemize]{noitemsep} % Make itemize lists more compact
    \makeatletter
    \newcommand*{\rom}[1]{\expandafter\@slowromancap\romannumeral #1@}
    \g@addto@macro{\UrlBreaks}{\UrlOrds}
    \makeatother

    \title{\LARGE \bf
    Optimization of Assignments for Teaching Assistants at UW
    }
    
    \author{ \parbox{3 in}{\centering Chongyi Xu, Weifan Jiang \\
             University of Washington\\
             MATH 381 Project Draft\\
             {\tt\small chongyix@uw.edu}\\{\tt\small wfjiang@uw.edu}}
    }

    \begin{document}
    \maketitle

    %----------------------------------------------------------------------------------------
    %	ARTICLE CONTENTS
    %----------------------------------------------------------------------------------------
    \begin{abstract}

    \indent For this project, we are focusing on developing a method to assign teaching-assistant candidates
    to different courses. Assigning candidates to their preferred teaching positions is important to 
    course coordinators at UW for a long time. Our purpose is to recommend a method of teaching-assistant
    assignments to solve this problem. We used simulated data to compare different methods that 
    we would use in the project.
        
    \end{abstract}

    \linespread{1.05} % Line spacing - Palatino needs more space between lines
    %------------------------------------------------
    \section{Problem Description}
    \indent The assignment of different teaching positions is a complicated task. The word "teaching position" 
    includes teaching assistants, graders and instructors here at UW. Each type of position has its 
    unique qualifications and requirments. Some positions require teaching, while other do not. 
    Most students are deemed to be certified to teach. Those whose first language is not English must 
    pass the SPEAK test to be certified. The position qualifications do not only appear in different roles 
    but also in different courses. For instance, the instructor positions would mostly be restricted to 
    graduate students or faculty. Meanwhile, the teaching assistant positions could open to both undergraduate students and 
    graduate students.
    \\ \indent The UW introduction courses to programming, such as CSE 142(Computer Programming I), are always popular. 
    There are over 700 students registered for around 50 sections each quarter. On the other hand, there are 
    also tiny-sized courses that designed for under 10 students. Therefore, the assignments of teaching positions 
    must satisfy the requirements for every single course each quarter such that everyone who registered for the course
    could have equal opportunity and fairly distributed teaching resources.   
    \\ \indent During the process of assignments, the course coordinators, who are in charge of assigning student candidates
    to appropriate roles, must consider the preference lists submitted by the candidates. Taking
    the example of UW CSE TA application form, as the candidates apply for CSE TA positions, they needs to choose their preferences 
    ("prefer not" or"neutral" or "prefer") for 12 distinct categories:
    \begin{itemize}
        \item AI and Robotics			
        \item Architecture			
        \item Computational Biology			
        \item Databases, Information Retrieval			
        \item Graphics, Vision, Animation			
        \item Hardware			
        \item Human-Computer Interaction			
        \item Introductory (CSE100,14x, 190)			
        \item Languages, Compilers, Software Engineering			
        \item Systems, Networks
        \item Theory			
        \item Uncategorize
    \end{itemize}
    \indent Following the information provided by CSE department, they initialize the preference value for each course that candidates prefer, 
    are neutral to, or prefer not to TA to 0.8, 0.5, and 0.2, respectively. This helps "push" candidates assignment towards courses 
    in areas they prefer and away from courses in areas they do not prefer. Without choosing preferences directly, candidates could establish
    their course preferences as well. If candidates choose to make up their own list, they would be asked to fill in a numerical number 
    between 0.0 and 1.0 that represents their preference to teach each courses that they are certified to TA. Besides preferences from 
    the candidates side, instructors preferences should also be considered. Instructors would be asked to fill in a form of preferred students.
    %------------------------------------------------
    \\ \indent Our motivation for the project is from the interview with undergraduate TAs and graduate TAs about their teaching experience in early quarters.
    (Hongtao Huang, hongth@cs.washington.edu, undergraduate teaching assistant at CSE; Tejas Devanur, tdevanur@uw.edu, graduate teaching fellow at Math Department
    ) They noticed that many times, even though they self-report their preferences, they got assigned to a course which 
    indicated as "less-preferred". Therefore, we would like to recommend a method that assigns candidates to
    courses, in such way that respects the following considerations:
    \begin{enumerate}
        \item Each candidate must be assigned to at most one course.
        \item Each course must be assigned an appropriate number of candidates.
        \item Each candidate must be assigned only to the courses for which they are qualified.
        \item Both candidates' and professors' preferences will be satisfied as much as possible.
    \end{enumerate}
    %------------------------------------------------
    \section{Simplification}
    In order to determine the best assignment that satisfy the above constraints, we will consider two things: what information is needed to
    find the optimal solution, and what assumption we need to make.
    \subsection{Input}
    \indent Below are the information needed to find the optimal solution:
    \begin{itemize}
        \item Preference of candidates for each role (grader, teaching assistant, instructor) for each course as numerical values between 0.0 and 
        1.0 where 0.0 indicates minimal preference and 1.0 indicates maximal preference.
        \item Qualification for each role for each course, which could be represented as an indicating matrix, where 1 entry indicates
        qualified for the role and 0 indicates not qualified.
        \item Preference of courses for each candidate to each role, which should also be a numerical value between 0.0 and 1.0 (larger indicates
        higher preference level).
        \item Capacity for each course (number of candidates needed for each role of any course, each course may differ).
    \end{itemize}
    %------------------------------------------------
    \subsection{Assumption}
    There were some assumptions we considered about in the draft phase
    \begin{enumerate}
        \item The quantifications of candidates who are applying for the same course were the same. In the other word, we would not take
        account into candidates' past teaching experiences, as well as their GPA when they took that course at this time.
        \item Student candidates do not care about any factors other than their preferences, such as payment and work time.
        \item Candidate's time conflict with other courses they are taking is not considered.
        \item All candidates are legally registered UW students.
    \end{enumerate}
    For this project, we will ignore the need of other roles, and only focus on teaching assistant role. 
    \section{Mathematical Model}
    Let $X = {x_1,...,x_m}$ represent $m$ student candidates, let $Y = {y_1,...,y_n}$ represent $n$ courses. \\
    Let $c_j$ represent the number of teaching assistants required for course $y_j$ for $1 \leq j \leq n$. \\
    Let $$q_{ij} = \begin{cases}1\text{, if $x_i$ is qualified to teach $y_j$} \\ 0\text{, otherwise} \end{cases}$$ \\
    The goal is to produce an assignment of candidates to courses, which should significantly consider the preference level of courses and candidates
    to each other. An assignment can be represented as 
    $$a_{ij} = \begin{cases}1\text{, if $x_i$ is assigned to $y_j$} \\ 0\text{, otherwise} \end{cases}\text{,}$$ 
    subject to the following hard constraints:
    \begin{enumerate}
        \item Each candidate must be assigned to at most one course: $$\forall x_i \in X, \sum_{j = 1}^n a_{ij} = 1\text{.}$$
        \item Each course must be assigned an appropriate number of candidates: $$\forall y_j \in Y, \sum_{i = 1}^m a_{ij} = c_j\text{.}$$.
        \item Each candidate must be assigned only to the courses for which they are qualified: $$\forall x_j \in X,\ \forall y_j \in Y q_{ij} \geq a_{ij}\text{.}$$.
    \end{enumerate}
    \section{Solution of the Mathematical Problem}
    \subsection{Stable Marriage Algorithm}
    \indent In the field of computer science and mathematics, the stable match problem or stable marraige problem states that given N men and N women, 
    where each person has ranked all members of the opposite sex in order of preference, marry the men and women together such that there are no 
    2 people of opposite sex who would both rather have each other than their current partners. If there are no such people, all the marriages are “stable”.
    \\ \indent In 1962, D. Gale and L. S. Shapley, proved that, for any equal number of men and women, it is guaranteed
    that there is a stable matching. In their paper "College Admission and the Stability of Marriage", they defined the stability as following, an assignment
    of applications to colleges will be called unstable if there are two applicants $\alpha$ and $\beta$ who are assigned to colleges A and B, respectively,
    although $\beta$ prefers A to B and A prefers $\beta$ to $\alpha$. They considered a stable assignment to be optimal if every applicant is at least 
    as well off under it as under any other stable assignments.
    \\ 
    \begin{lstlisting}
*Gale-Shapley Algorithm*
INPUT: preference list for men and 
women
INITIALIZE matching set S to an empty 
set
WHILE (some woman w in W is still 
    unmatched and hasn't proposed 
    to every man in M)
    m <- first man on w's preference 
        list to whom w has not yet 
        proposed
    IF (m is unmatched)
        ADD pair (m, w) to S
    ELSE IF (m prefers w to existing 
            pair w')
        REPLACE (m, w') with (m, w) 
        FREE w'
    ELSE 
        w REJECT m
RETURN: matching S
    \end{lstlisting}
    \indent In this project, we have to slightly modify the algorithm in order to achieve our goal. Since each course may have need 
    of more than one candidate to be assigned, such changes will be made to the original Gale-Shapley Algorithm:
    \begin{enumerate}
        \item During each round of proposing, a currently unmatched candidate proposes to his/her top-choice course which he/she 
        has not proposed to yet.
        \item After candidates finish proposing to courses, each course takes the new proposers, put them into the same "set" with
        other candidates that are already matched with this course, to form a "temporary" waitlist.
        \item If the waitlist's length exceeds the course capacity, the waitlist will be sorted by course's preference to waitlist's
        members, and only the top $k$ ones will be kept, with $k$ being the capacity of that course.
        \item The algorithm terminates when there are no unmatched candidates or all candidates have proposed to all courses.
    \end{enumerate}
    \subsection{Hungarian Algorithm}
    The Hungarian Algorithm is a combinatorial optimization algorithm that solves the assignment problem in polynomial time.
    It was developed and published in 1955 by Harold Kuhn, who gave "Hungarian Algorithm" its name according to the previous works
    of two Hungarian mathematicians.
    \begin{lstlisting}
*Hungarian Algorithm*
INPUT: n*n cost matrix A
FOR EACH (row R_A in A)
    SUBTRACT min(R_A) from R_A
FOR EACH (column C_A in A)
    SUBTRACT min(C_A) from C_A
LABEL appropriate entries so that all
      zero entries are covered and 
      minimum number of labels are 
      used
IF (# labels = n)
    RETURN: labels as assignment    
ELSE:
    SUBTRACT min(A) from unlabeled 
              R_A
    ADD min(A) to unlabeled C_A
    REPEAT from LABEL
    \end{lstlisting}
    For this problem, the original Hungatian Algorithm has to be modified to be compatible with this problem:
    \begin{enumerate}
        \item The result of this problem is a many-to-one matching (multiple candidates assigned to one course). In order to 
        convert the problem to one-to-one matching, we will be splitting each course into slots (for example, if course A
        requires 5 candidates to be assigned, we will have 5 "slots" from A1 to A5, to corresponds to the 5 candidates wanted
        by A).
        \item Hungarian Algorithm also needs to run on a square matrix. We assume that there will always be more candidates than
        slots (if not, the department will need to advertise more to get more student apply as candidate). Thus, we can add
        "dummy" slots to make number of candidates and number of slots equal to each other. If a candidate matches to one of the
        dummy slots, this candidate is unselected for the row of teaching assistant.
        \item A "cost matrix" needs to be constructed for Hungarian Algorithm, and optimal solution (which is the output of
        Hungarian Algorithm) has minimized total cost. We let the row of cost matrix to be candidates, and column be the slots.
        Therefore, the value of $(i. j)$ should be:
        \begin{itemize}
            \item if $j$ is a "dummy" slot, then $(i, j)$ should be $0$ regardless of $i$. We need all "dummy" slots to have
            the same value, so the optimality if all "non-dummy" matches are not influenced by the dummy variables.
            \item if $i$ is not qualified to teach $j$, the value of $(i, j)$ should be infinity, therefore the Hungarian algorithm
            will avoid large cost and not choosing the unqualified entry. If the output assignment's cost is infinity, it indicates
            that no possible matching is available.
            \item If $i$ qualified to teach $j$, the $(i, j)$ entry should be $2 - \text{i's preference to j} - \text{j's preference to i}$
            therefore the cost is smaller if the sum of preference of candidate and course to each other is larger.
        \end{itemize}
    \end{enumerate}
    After making such modifications, Hungarian's algorithm will output an assignment of candidates to slots, which can be
    transferred to an assignment of candidates to courses. The sum of preferences to each other for all matched pair of
    candidates and courses is maximized.
    \subsection{Maximum Matching Algorithm}
    Consider an undirected graph $G=(V,E)$. A matching M is said to be maximal if M is not properly contained in any other matching.
    Formally, $M\notin M^{'}$ for any matching $M^{'}$ of $G$. Intuitively, this is equivalent to saying that a matching is maximal 
    if we cannot add any edge to the existing set. And a matching $M$ is said to be Maximum if for any other matching $M^{'}$, 
    $|M|\geq |M^{'}|$. Generally, maximum matching applied to unweighted graph more but for this project,
    we would like to modify the algorithm with weights in order to meet our propose. With researching, we decided to implement the method
    introduced by \textit{Zvi Galil}, Department of Computer Science, Columbia University, in 1986. In his study "Efficient Algorithms
    for Maximum Matching in Graphs", he developed this method based on Berge's Theorem, "the matching M has maximum cardinality if and 
    only if there is no augmenting path with respect to M."

    \indent Given a graph, $G=(V,E)$ and a matching $M \subset E$, a path $P$ is called an augmenting path for $M$ if:
    \begin{itemize}
        \item The two end points of $P$ are unmatched by $M$
        \item The edges of P alternate between edges $\in M$ and edges $\notin M$.
    \end{itemize}
    
    \begin{lstlisting}
*Maximum Matching*
INPUT: Graph G
M <- random selected matching
WHILE (there is a blossom and there 
       is an augmenting path in M)
    GROW the forest, labeling the 
         vertices even/odd
    IF (there is a blossom in the 
        graph)
        SHRINK the blossom to obtain 
               a new graph G'
        CONTINUE foresting
    ELSE
        FIND such even - even edges 
             to obtain a maximally 
             disjoint set of 
             augmenting paths 
             (P1,...,Pk)
    M <- switching edges along P's 
         from in-to-out of M and 
         vice-versa
EXPAND all blossoms to obtain the 
       maximum matching in the 
       original graph G.

    \end{lstlisting}
    \section{Evaluation of methods}
    \subsection{Scoring of assignments}
    In order to compare and contrast each algorithm discussed above, we will develop a "score function", which takes input
    of a produced assignment of candidates and courses, and output a numerical score, which higher score indicates better
    quality of the assignment.\\
    The input of the scoring function should be a serie of binary variables:
    $$\sigma_{i,j}=\begin{cases}1\text{, if candidate i is assigned to course j} \\
    0\text{, else}\end{cases}$$
    for each candidate $i$ and each course $j$ in the assignment.\\
    Let $m_{i, j}$ be candidate $i$'s preference score to course $j$, and $n_{j, i}$ be course $j$'s preference to candidate $i$,
    both $m$ and $n$ have value between $0$ and $1$. The return value of score function should be:
    $$\sum_i\sum_j \sigma_{i,j}(\lambda_1m_{i,j} + \lambda_2n_{j,i})$$.
    Which $\lambda_1$ and $\lambda_2$ are different weights we consider course's and candidates's preferences to the other. Currently,
    we use $1$ for both weights to value candidates and courses' opinions equally when scoring an assignment.
    All invalid assignments which breaks any of rule 1, 2, 3 at the end of section "Abstract" does not receive a score and should not be
    considered. (Theoretically, each algorithm described above should avoid generating an invalid assignment.)

    \subsection{Additional Metrics}
    In addition to sum the numerical preferences in the assignment, we also measure the percentage of courses and candidates which has their top-3
    requests satisfied. In more detail, in an assignment, for a matched pair (a, b), which a is candidate and b is course:
    \begin{itemize}
        \item if a's preference score for b is within the top 3 scores of all a's preference scores, we consider b is a "top-3" choice for a. Same logic
        applied to b when checking if a is a "top-3" choices for a.
        \item If there are ties: suppose there are five courses: c1, c2, c3, c4, c5, and candidate a's preference score to these courses are: 0.9, 0.8, 0.8, 0.7,
        0.6 in order, then the top 3 scores should be: 0.9, 0.8, 0.7. In this case c1, c2, c3, c4 all satisfy as "top-3" choices.
    \end{itemize}
    After getting the "top-3" satisfication count for both candidates and courses, we calculate the satisfiacation rate by:
    $$\frac{\text{number of matches satisfying "top-3"}}{\text{all matches}} \times 100\%\text{,}$$
    and "all matches" should be sum of capacity (number of TA wanted) of all courses.
    \\ We will use both the score, and preference satisfication rate of courses and candidates, to evaluate each method, and determine the best one.

    \section{Results}
    \subsection{Simulations}
    By using \verb|Python|, we implemented the following functionalities. 
    By inputting a \verb|num_candidate| and a \verb|num_course| parameters, our program can:
    \begin{enumerate}
        \item Generate $m$ candidates and $n$ courses, which \verb|num_candidate = m| and \verb|num_course = n|.
        \item Generate random preference levels of each candidate for courses, and each course to candidates, which are
        floats between $0$ and $1$ inclusively.
        \item Generate capacity for each course, which is an integer between $1$ and $5$ inclusively.
        \item Generate qualification for each student for each course, which is an integer either $0$ or $1$, with $1$ indicates
        being qualified.
        \item Run the generated data on one (or all) of the three algorithms mentioned above.
    \end{enumerate}
    Note: for the stable-marriages method, we coded the algorithm as described in an article from Cornell University; for the Hungarian Algorithm, we
    constructed the cost matrix as defined in section 4.2, and used the \verb|scipy.optimize.linear_sum_assignment| package which exactly
    performs the Hungarian Algorithm; for the maximal matching algorithm, we used the \verb|networkx.max_weight_matching| package.
    \\ For each set of random data (includes course preference, candidate preference, capacity and qualifications) generated, we will run
    all three algorithms on it, and record the evaluations (score, preference satisfication rate for both sides) for each method.

    \subsection{Simulation Result}

    \section{Conclusion}
    \indent From the output, we could draw our conclusion that our method can build an appropriate TA assignment that satisfies candidates' preferences
    and Instructors' preferences as much as possible, while all requirements are met.

    \section{Improvements}
    We are currently only focusing on teaching assistant posistions, but the real world problem includes more various teaching
    positions such as graders and instructors. We would also like to consider various positions for further studies.
    \\ \indent Currently we value course's preference to candidates and candidates' preference to courses equally likely. We will
    test additional weights of them. This can be achieved by adjusting computation of the cost matrix in Hungarian Algorithm, and
    changed the direction of proposing in the Stable Marriage Algorithm, and change the value of $\lambda_1$ and $\lambda_2$ in the
    scoring function. We want to know the real-world implications if different weights are used.
    \\ \indent Another point to improve our method is to find out some other constraints to let our model fit the real world problem more. For example,
    we could add the constraints considering about the time confliction for candidate's course schedule. 
    \\ \indent In such way, I think our model could be more realistic and might be more accpetable to our community partners.

    \section{References}
    [1]"The college admission problem: many-to-one matching : Networks II Course blog for INFO 4220", Blogs.cornell.edu, 2018. [Online]. 
    Available: https://blogs.cornell.edu/info4220/2016/03/18/the-college-admission-problem-many-to-one-matching/. \\

    [2]D. Gale and L. Shapley, "College Admissions and the Stability of Marriage", The American Mathematical Monthly, vol. 69, no. 1, p. 9, 1962. \\
    
    [3]"Stable Marriage Problem -- from Wolfram MathWorld", Mathworld.wolfram.com, 2018. http://mathworld.wolfram.com/StableMarriageProblem.html. \\
    
    [4]Cs.princeton.edu, 2018. [Online]. https://www.cs.princeton.edu/~wayne/kleinberg-tardos/pdf/01StableMatching.pdf. \\
    

\end{document}